\documentclass[11pt]{article}

\usepackage{amsmath} % Required for \eqref
\usepackage{amssymb} % Required for \mathbb
\usepackage{units}   % Required for \nicefrac
\usepackage{float}   % Required for algorithm and floating environments
\usepackage{graphicx}% Required for figures and imagesf

%-------------------------------------------
\newtheorem{theorem}{Theorem}
\newtheorem{lemma}[theorem]{Lemma}
\newtheorem{acknowledgment}[theorem]{Acknowledgment}
\newtheorem{proposition}[theorem]{Proposition}
\newtheorem{corollary}[theorem]{Corollary}
%-------------------------------------------
\floatstyle{ruled}
\newfloat{program}{thp}{lop}
\floatname{program}{Algorithm}
%-------------------------------------------
\newcommand{\diag}{\ensuremath{\mathrm{diag}}}
\newcommand{\MSE}[1]{\ensuremath{\mathrm{MSE}\left(#1\right)}}
\newcommand{\trace}[1]{\ensuremath{\mathrm{trace}\left( #1 \right)}}
\newcommand{\norm}[1]{\ensuremath{\left\|#1\right\|_2^2}}
\newcommand{\func}[2]{\ensuremath{\mathrm{#1}\left( #2 \right)}}
\newcommand{\normzero}[1]{\ensuremath{\left\|#1\right\|_0}}
\newcommand\eps \epsilon
\newcommand{\R}{\ensuremath{\mathbb{R}}}
\newcommand\RR[1]{\mathbb{R}^{#1}}
\newcommand{\N}{\ensuremath{\mathbb{N}}}
\newcommand{\real}{\ensuremath{\mathbb{R}}}
\newcommand{\conv}{\ensuremath{\ast}}
\newcommand{\cl}[1]{\ensuremath{\mathcal{#1}}}
\newcommand{\suppsize}[1]{\ensuremath{|\mathcal{#1}|}}
\newcommand{\vect}[1]{\ensuremath{\mathbf{#1}}}
\newcommand{\matr}[1]{\ensuremath{\mathbf{#1}}}
\usepackage{amsfonts}
\usepackage{amsmath}
\usepackage{amssymb}
\usepackage{graphicx}
\usepackage{mathrsfs}

\providecommand\Laplacian{\nabla^2}
\providecommand\bnabla{\boldsymbol{\nabla}}
\providecommand\bcdot{\boldsymbol{\cdot}}
\providecommand\bv{\boldsymbol{v}}
\providecommand\bV{\boldsymbol{V}}
\providecommand\be{\boldsymbol{\hat{e}}}
\providecommand\bn{\boldsymbol{\hat{n}}}
\providecommand\bk{\boldsymbol{\hat{k}}}
\providecommand\bj{\boldsymbol{j}}
\providecommand\bi{\boldsymbol{i}}
\providecommand\bI{\boldsymbol{I}}
\providecommand\bJ{\boldsymbol{J}}
\providecommand\bx{\boldsymbol{x}}
\providecommand\bzero{\boldsymbol{0}}

\providecommand\tI{\mathsf{I}}
\providecommand\tS{\mathsfbi{S}}
\providecommand\unit{\boldsymbol{\hat{\imath}}}

\providecommand\dd{\mathrm{d}}
\providecommand\ee{\mathrm{e}}

\setlength{\textheight}{8.7in}
\setlength{\columnsep}{2.0pc}
\setlength{\textwidth}{6.6in}
% \setlength{\footheight}{0.2in}
\setlength{\topmargin}{0.05in}
\setlength{\headheight}{0.2in}
\setlength{\headsep}{0.1in}
\setlength{\evensidemargin}{0in}
\setlength{\oddsidemargin}{0in}
% \setlength{\parindent}{1pc} 
\setlength{\parindent}{0.0 in}
\setlength{\parskip}{0.1 in} 
\title{Notes on numerical methods for PDEs}
\author{Roman Zeyde}

\begin{document}
\maketitle
\section{PDEs}
\subsection{Laplace equation (follows from Gauss' law)}
The electric potential satisfies the elliptic equation
\begin{equation}
	\bnabla \bcdot (C \bnabla \varPhi) = 0; \label{Phi eqn}
\end{equation}
$C$ is salt concentration, playing the role of $\eps$, 
in Maxwell's equation (where $\vect{E} = -\bnabla \varPhi$):
\begin{equation}
\bnabla \bcdot \eps \vect{E} = \rho_{free}
\end{equation} 

We assume regular grid $\{(x_i, y_j)\}$,
where $i\in\{0 \ldots n\}$ and $j\in\{0 \ldots m\}$. The maximal and the
minimal index' values correspond to the boundary of the problem.

We express the differential operator as:
\begin{equation}
C_{i+1/2,j} \cdot \frac{\varPhi_{i+1,j}-\varPhi_{i,j}}{x_{i+1}-x_{i}} - 
C_{i-1/2,j} \cdot \frac{\varPhi_{i,j}-\varPhi_{i-1,j}}{x_{i}-x_{i-1}} +
C_{i,j+1/2} \cdot \frac{\varPhi_{i,j+1}-\varPhi_{i,j}}{y_{i+1}-y_{i}} - 
C_{i,j-1/2} \cdot \frac{\varPhi_{i,j}-\varPhi_{i,j-1}}{y_{i}-y_{i-1}}
\end{equation}
where $C_{i\pm 1/2,j} = (C_{i,j} + C_{i\pm 1,j})/2$ and 
$C_{i,j\pm 1/2} = (C_{i,j} + C_{i,j\pm 1})/2$.
\subsection{Diffusion--advection equation}
The salt concentration $C$ is governed by the 
\begin{equation}
	\Laplacian C  = \alpha \, \bV \bcdot \bnabla C ; \label{C eqn}
\end{equation}
\section{Iterative solver}
\end{document}