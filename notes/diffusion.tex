\section{Diffusion}
\subsection{Laplace equation (follows from Gauss' law)}
The electric potential satisfies the elliptic equation
\begin{equation}
	\bnabla \bcdot (C \bnabla \varPhi) = 0; \label{Phi eqn}
\end{equation}
$C$ is salt concentration, playing the role of $\eps$,
in Maxwell's equation (where $\vect{E} = -\bnabla \varPhi$):
\begin{equation}
\bnabla \bcdot \vect{D} = \bnabla \bcdot \eps \vect{E} = \rho_{free}
\end{equation}

We assume regular grid $\{(x_i, y_j)\}$,
where $i\in\{0 \ldots n\}$ and $j\in\{0 \ldots m\}$. The maximal and the
minimal indices' values correspond to the boundary of the problem.

We approximate the differential operator as:
\begin{eqnarray}
\left[ \bnabla \bcdot (C \bnabla \varPhi) \right]_{i,j}
&\approx &\frac{2}{x_{i+1,j} - x_{i-1,j}} \left(
C_{i+1/2,j} \cdot \frac{\varPhi_{i+1,j}-\varPhi_{i,j}}{x_{i+1}-x_{i}} -
C_{i-1/2,j} \cdot \frac{\varPhi_{i,j}-\varPhi_{i-1,j}}{x_{i}-x_{i-1}}
\right) + \\
&&\frac{2}{y_{i,j+1} - y_{i,j-1}} \left(C_{i,j+1/2} \cdot \frac{\varPhi_{i,j+1}-\varPhi_{i,j}}{y_{i+1}-y_{i}} -
C_{i,j-1/2} \cdot \frac{\varPhi_{i,j}-\varPhi_{i,j-1}}{y_{i}-y_{i-1}}
\right)
\end{eqnarray}
where $2C_{i\pm 1/2,j} = C_{i,j} + C_{i\pm 1,j}$ and
$2C_{i,j\pm 1/2} = C_{i,j} + C_{i,j\pm 1}$.

The algebraic system above can be written as $\matr{A} \vect{u} = \vect{f}$,
where $\matr{A}$ depends on the grid and current $C$ values.
$\vect{f}$ can be chosen to be any function, which is useful for testing the solver.
We initialize $\vect{u}$ to any function, and employ Jacobi iterations
(by writing $\matr{A} = \matr{D} + (\matr{A-D})$, where $\matr{D}$ is diagonal):
\begin{eqnarray}
\vect{f} &=& \matr{A}\vect{u} = \matr{D}\vect{u} + (\matr{A-D})\vect{u} \\
\matr{D}\vect{u} &=& \vect{f} - (\matr{A-D})\vect{u} =
\vect{f} - \matr{A}\vect{u} + \matr{D}\vect{u} \\
\vect{u}_+ &=& \vect{u} + \matr{D}^{-1}(\vect{f} - \matr{A}\vect{u}) =
(\matr{I} - \matr{D}^{-1}\matr{A}) \vect{u} + \matr{D}^{-1}\vect{f} \\
\end{eqnarray}
Therefore, we shall use the following iteration scheme:
\begin{eqnarray}
\vect{u}_+ &=& \matr{T}\vect{u} + \vect{d} \\
\matr{T} &=& \matr{I} - \matr{D}^{-1}\matr{A} \\
\vect{d} &=& \matr{D}^{-1}\vect{f}
\end{eqnarray}
To speed-up the convergence, we shall use Red-Black Gauss-Seidel method,
that updates ``odd'' and ``even'' elements of $\vect{u}$, solving
effectively the linear system for half the variables each time.
\subsection{Boundary conditions}
\subsubsection{Dirichlet}
The condition is given as $U(x, y) = C$.
We take $u_k = U(x_i, y_j) = C$.
Now, it is simply substituted and
linear system's right-hand side is updated accordingly.

\subsubsection{Neumann}
The condition is given as $\bnabla U(x,y) \cdot \vect{\hat n} = C$,
where $\vect{\hat n}$ is normal to domain's boundary.
If we use quadratic approximation, the gradient can be expressed as:
\begin{eqnarray}
  f(x) &=& ax^2 + bx + c \\
  f(x_0 \pm \Delta x) &=&
   a(x_0 \pm \Delta x)^2 + b(x_0 \pm \Delta x) + c \\
  \frac{f(x_0 + \Delta x) - f(x_0 - \Delta x)}{2\Delta x} &=&
   2a x_0 + b = f'(x_0) \\
\end{eqnarray}
Therefore, the boundary value can be approximated by:
\begin{eqnarray}
 U(x + \Delta x,y + \Delta y) &\approx& U(x, y) + \partial_x U(x_c,y_c) \Delta x +
 \partial_x U(x_c,y_c) \Delta y \\
 x_c &=& x + \frac{\Delta x}{2} \\
 y_c &=& y + \frac{\Delta y}{2}
\end{eqnarray}

\subsection{Specific boundary conditions}
\subsubsection{$R = 1$}
\begin{eqnarray}
  \frac{\partial C}{\partial R} &=&
   C \frac{\partial \varPhi}{\partial R} \\
  \varPhi &=& \mathscr{V} - \ln C + \ln \gamma
\end{eqnarray}
Take $\mathscr{V} = -\ln \gamma$ and note that:
\begin{eqnarray}
  \frac{\partial}{\partial R} \ln C &=&
  \frac{1}{C}\frac{\partial C}{\partial R} =
  \frac{\partial \varPhi}{\partial R} \\
  \varPhi &=& -\ln C
\end{eqnarray}
Use ghost-point \#0 to write the following boundary equations:
\begin{eqnarray}
  \varPhi_1 - \varPhi_0 &=& \ln C_1 - \ln C_0 \\
  \varPhi_1 + \varPhi_0 &=& - \ln C_1 - \ln C_0
\end{eqnarray}
Add and subtract to get 2 Dirichlet conditions:
\begin{eqnarray}
  \varPhi_1 &=& - \ln C_0 \\
  \varPhi_0 &=& - \ln C_1 \\
  C_0 &=& e^{-\varPhi_1}
\end{eqnarray}
\subsubsection{$R \rightarrow \infty$}
$C$ gets uniform, and we get Dirichlet:
\begin{eqnarray}
 C = 1
\end{eqnarray}
The field radial component is Neumann:
\begin{eqnarray}
 \frac{\partial \varPhi}{\partial R} = -\mathscr{E} \cos \theta
\end{eqnarray}
\subsubsection{$\Theta = 0/\pi$}
Using symmetry considerations, we have Neumann conditions:
\begin{eqnarray}
 \frac{\partial \varPhi}{\partial \Theta} &=& 0 \\
 \frac{\partial C}{\partial \Theta} &=& 0 \\
\end{eqnarray}

