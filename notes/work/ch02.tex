\section{Numerical Solver}

%%%%%%%%%%%%%%%%%%%%%%%%%%%%%%%%%%%%%%%%%%%%%%%%%%%%%%%%%%%%%%%%%%%%%%%%%%%%%%%%

\subsection{Discretization}
\subsubsection{Spherical Coordinates}
Because the system is axisymmetric, the operators are written in 
spherical coordinates $(r,\theta,\phi)$ 
(using the derivation of Appendix \ref{append:spherical}).

Due to axial symmetry, 
the scalar gradient and the divergence can be written as:
\begin{eqnarray}
\bnabla f &=& \deriv{f}{r}\brhat + \frac{1}{r}\deriv{f}{\theta}\bthetahat, \\
\bnabla \cdot \bF &=& \frac{1}{r^2}\deriv{}{r}\pars{F_r r^2 } + 
               \frac{1}{r \sin\theta}\deriv{}{\theta}\pars{F_\theta \sin\theta}.
\end{eqnarray}
The scalar Laplacian can be written as:
\begin{eqnarray}
\Laplacian f = \bnabla \cdot (\bnabla f)&=& 
 \frac{1}{r^2}\deriv{}{r}\pars{r^2 \deriv{f}{r}} + 
 \frac{1}{r^2 \sin\theta}\deriv{}{\theta}\pars{\sin\theta \deriv{f}{\theta}}.
\end{eqnarray}
The vector gradient can be written as:
\begin{eqnarray}
\bnabla \bF &=& \deriv{F_r}{r} \brhat \brhat + \deriv{F_\theta}{r} \brhat \bthetahat + 
\frac{1}{r}\pars{\deriv{F_r}{\theta} - F_\theta} \bthetahat \brhat + 
\frac{1}{r}\pars{\deriv{F_\theta}{\theta} + F_r} \bthetahat \bthetahat.
\end{eqnarray}
The vector Laplacian can be written as:
\begin{eqnarray}
\bLaplacian \bF &=& 
\left(\Laplacian F_r - \frac{2F_r}{r^2} - 
\frac{2}{r^2 \sin\theta} \deriv{\left(F_\theta \sin\theta \right)}{\theta}\right)\brhat
+ \left(\Laplacian F_\theta - \frac{F_\theta}{r^2 \sin^2\theta} + 
\frac{2}{r^2}\deriv{F_r}{\theta}\right) \bthetahat.
\end{eqnarray}

\subsubsection{Computational Grid}
In order to discretize the differential operators, 
a regular grid of size $n_r \times n_\theta$ is defined, with:
\begin{eqnarray}
(r_i,\theta_j) &\in& [1, \infty) \times [0,\pi], \\ 
r_i &=& (1+\Delta_r)^i, \\
\theta_j &=& \Delta_\theta \cdot j,
\end{eqnarray}
where a logarithmic grid spacing is used for $r$ and a uniform grid is used for $\theta$.
Note that $r_0 = 1$ and $R_{max} = (1+\Delta_r)^{n_r}$ should be much 
larger than $r_0$. Thus:
\begin{eqnarray}
\Delta_r &=& \pars{R_{max}} ^ \frac{1}{n_r} - 1, \\
\Delta_\theta &=& \frac{\pi}{n_\theta}.
\end{eqnarray}

This grid induces a disjoint subdivision of the domain 
$\Omega = [1, R_{max}] \times [0,\pi]$ into cells:
\begin{eqnarray}
\bigcup_{ij}\Omega_{ij} &=& \Omega.
\end{eqnarray}
A specific cell $\Omega_{ij}$ and its center $(\bar{r}_i, \bar{\theta}_j)$ are defined by:
\begin{eqnarray}
\Omega_{ij} &=& [r_{i-1}, r_{i}] \times [\theta_{j-1}, \theta_{j}], \\
\bar{r}_i &=& \frac{r_{i-1} + r_{i}}{2}, \\
\bar{\theta}_j &=& \frac{\theta_{j-1} + \theta_{j}}{2}.
\end{eqnarray}

Each variable is discretized with respect of its cell as follows:
\begin{itemize}
\item $\varPhi$, $C$ and $P$ are represented by their value at the center of each cell, 
using an all-centered grid:
\begin{eqnarray}
\varPhi^h[i,j] &=& \varPhi(\bar{r}_i, \bar{\theta}_j), \\
C^h[i,j] &=& C(\bar{r}_i, \bar{\theta}_j), \\
P^h[i,j] &=& P(\bar{r}_i, \bar{\theta}_j).
\end{eqnarray}
\item $\bV$ is represented by its values at cell boundaries, using a staggered grid:
\begin{eqnarray}
V_r^h[i,j] &=& V_r(r_i, \bar{\theta}_j), \\
V_\theta^h[i,j] &=& V_\theta(\bar{r}_i, {\theta}_j).
\end{eqnarray}
\end{itemize}

\subsubsection{Operator Discretization}
A finite-volume method with linear interpolation is used for flux 
$\boldsymbol{f}^h$ discretization. 

Define the following discrete central difference operators:
\begin{eqnarray}
\cD_r(f^h)[i,j] &=& \frac{f^h\left[i+\half,j\right] - f^h\left[i-\half,j\right]}
                       {r\left[i+\half,j\right] - r\left[i-\half,j\right]}, \\
\cD_\theta(f^h)[i,j] &=& \frac{f^h\left[i,j+\half\right] - f^h\left[i,j-\half\right]}
					   {\theta\left[i,j+\half\right] - \theta\left[i,j-\half\right]}.
\end{eqnarray}

Define the following interpolation operators:
\begin{eqnarray}
\cI_r(f^h)[i,j] &=& \frac{
\pars{r\left[i,j\right] - r\left[i-\half,j\right]} 
  f^h\left[i+\half,j\right] + 
\pars{r\left[i+\half,j\right] - r\left[i,j\right]} 
  f^h\left[i-\half,j\right] 
}{r\left[i+\half,j\right] - r\left[i-\half,j\right]},
\\
\cI_\theta(f^h)[i,j] &=& 
\frac{
\pars{r\left[i,j\right] - r\left[i,j-\half\right]} 
  f^h\left[i,j+\half\right] + 
\pars{r\left[i,j+\half\right] - r\left[i,j\right]} 
  f^h\left[i,j-\half\right] 
}{r\left[i,j+\half\right] - r\left[i,j-\half\right]}.
\end{eqnarray}

The total flux of a cell is equal to zero (due to conservation):
\begin{eqnarray}
\bnabla \cdot \boldsymbol{f}^h &=& 0, 
\\
\frac{1}{r^2} \cD_r\pars{f^h_r r^2} + 
\frac{1}{r \sin\theta} \cD_\theta\pars{f^h_\theta \sin\theta} &=& 0, 
\\
\cD_r\pars{f^h_r \cdot r^2 \sin\theta} + 
\cD_\theta\pars{f^h_\theta \cdot r \sin\theta} &=& 0. 
\end{eqnarray}

Ion fluxes are discretized on grid cell boundaries:
\begin{eqnarray}
I^h_r &=& -\cI_r(C^h) \cdot \cD_r(\varPhi^h), \\
I^h_\theta &=& -\cI_\theta(C^h) \cdot \frac{\cD_\theta(\varPhi^h)}{r}.
\end{eqnarray}

Salt fluxes are discretized on grid cell boundaries 
(using upwind scheme $\mathcal{U}$ for numerical stability at large cell Peclet number):
\begin{eqnarray}
J^h_r &=& -\cD_r(C^h) + \alpha V^h_r \cdot \mathcal{U}^{\bV^h}_r (C^h), \\
J^h_\theta &=& -\frac{\cD_\theta(C^h)}{r} + \alpha V^h_\theta \cdot \mathcal{U}^{\bV^h}_\theta (C^h), \\
 \mathcal{U}^{\bV^h}_r(C^h)[i,j] &=& C^h\left[i-\frac{\sign(V^h_r)}{2}, j\right], \\
 \mathcal{U}^{\bV^h}_\theta(C^h)[i,j] &=& C^h\left[i, j-\frac{\sign(V^h_\theta)}{2}\right]. 
\end{eqnarray}

Mass flux $\bV$ is discretized on grid cells boundaries, 
using the velocity staggered grid.

Force components are discretized on the velocity staggered grid, where
linear interpolation is used for the Coulomb force.
\begin{eqnarray}
F^h_r &=& -\cD_r(P^h) 
          + \cL(V^h_r) - \frac{2}{r^2} V^h_r 
		  - \frac{2}{r^2 \sin\theta} \cI_r(\cD_\theta (V^h_\theta \sin\theta))
          + \cD_r(\varPhi^h) \cdot \cI_r(\cL(\varPhi^h)), \\
F^h_\theta &=& -\frac{\cD_\theta(P^h)}{r} 
		  + \cL(V^h_\theta) - \frac{F^h_\theta}{r^2 \sin^2\theta} 
		  + \frac{2}{r^2} \cI_\theta(\cD_\theta(F^h_r))
		  + \frac{\cD_\theta(\varPhi^h)}{r} \cdot \cI_\theta(\cL(\varPhi^h)), \\
\cL(f^h) &=& \frac{1}{r^2}\cD_r\pars{\cD_r(f^h) r^2} + 
\frac{1}{r^2 \sin\theta} \cD_\theta\pars{\cD_\theta(f^h) \cdot \sin\theta}.
\end{eqnarray}


\subsubsection{Boundary Conditions}
In order to discretize boundary condtions, 
``ghost'' points are employed. 
The grid is extended to include points outside the domain interior,
and the variables at the ``ghost'' points are set to satisfy 
the discretized boundary conditions, following the analysis below.

The boundary conditions at the ion-exchange boundary ($r=1$) are:
\begin{eqnarray}
\cI_r(\varPhi^h) &=& -\cI_r(\log C^h), \\
\cD_r(\varPhi^h) &=& \cD_r(\log C^h), \\
V^h_r &=& 0, \\
V^h_\theta &=& 4\log\pars{\frac{1 + \exp\left\{\cI_\theta(\zeta^h)/2\right\}}{2}} \cdot 
			\cD_\theta(\zeta^h), \\
  \zeta^h &=& - \log \gamma - \cI_r(\varPhi^h).
\end{eqnarray}
The equations are discretized using ``ghost'' points outside the interior of the domain:
\begin{eqnarray}
 \varPhi^h[1/2, j] = \frac{\varPhi^h[0,j] + \varPhi^h[1,j]}{2} &=& 
	-\frac{\log C^h[0,j] + \log C^h[1,j]}{2} = -\log C^h[1/2, j], \\
  \frac{\varPhi^h[1,j] - \varPhi^h[0,j]}{\Delta r} &=& 
	\frac{\log C^h[1,j] - \log C^h[0,j]}{\Delta r}. 
\end{eqnarray}
By adding and subtracting the equations, we have:
\begin{eqnarray}
\varPhi^h[0,j] &=& -\log\brcs{C^h[1,j]}, \\
C^h[0,j] &=& \exp\brcs{-\varPhi^h[1,j]}, \\
  \zeta^h[j] &=& \cV - \varPhi^h[1/2,j]
                     = -\log\gamma -\frac{\varPhi^h[0,j] + \varPhi^h[1,j]}{2}.
\end{eqnarray}
The slip condition can be written as:
\begin{eqnarray}
V_r^h[0,j] &=& 0, \\
V_\theta^h[1/2,j] &=& 
4\log\pars{\frac{1 + \exp\left\{\frac{1}{2}
 \frac{\zeta^h[j] + \zeta^h[j+1]}{2}\right\}}{2}} 
\cdot \frac{\zeta^h[j+1] - \zeta^h[j]}{\bar{\theta}_{j+1} - \bar{\theta}_{j}}, \\
  V^h_\theta[0,j] &=& 2 V^h_\theta[1/2, j] - V^h_\theta[1,j].
\end{eqnarray}

For $\theta = 0$ and $\theta = \pi$, the ``ghost'' points are defined by:
\begin{eqnarray} 
\brc{l}{
\varPhi^h[i, 0] = \varPhi^h[i, 1] \\
C^h[i, 0] = C^h[i, 1] \\
V_r^h[i, 0] = V_r^h[i, 1] \\
V_\theta^h[i, 0] = 0
} 
\brc{l}{
\varPhi^h[i, n_\theta+1] = \varPhi^h[i, n_\theta], \\
C^h[i, n_\theta+1] = C^h[i, n_\theta], \\
V_r^h[i, n_\theta+1] = V_r^h[i, n_\theta], \\
V_\theta^h[i, n_\theta+1] = 0.
} 
\end{eqnarray}

Far away from the ion-exchange boundary ($r\rightarrow\infty$) we have:
\begin{eqnarray}
\frac{\varPhi^h[n_r + 1, j] - \varPhi^h[n_r, j]}{\bar{r}_{n_r + 1} - \bar{r}_{n_r}}, 
 & = & -\beta \cos\bar{\theta}_j, \\
C^h[n_r + 1, j] & = & 1 \\
V_r^h[n_r + 1, j] & = & -\cU \cos\bar{\theta}_j, \\
V_\theta^h[n_r + 1, j] & = & \cU \sin\theta_j.
\end{eqnarray}

Because the actual grid is finite, the choice 
$r_{n_r} = R_{max}$ must be taken large enough so as to have a negligible effect
on the solution.

Note that $P^h$ has no boundary conditions -- so the pressure variable is defined 
only in the domain interior.

%%%%%%%%%%%%%%%%%%%%%%%%%%%%%%%%%%%%%%%%%%%%%%%%%%%%%%%%%%%%%%%%%%%%%%%%%%%%%%%%
\subsection{Solver Design}
\subsubsection{Newton's Method}
An operator $\cO$ is defined by its input vector $\bx$ and its output 
$\by = \cO(\bx)$.

In order to solve the equation $\cO(\bx) = \bzero$, Newton's method can 
be applied, given an initial solution $\bx_0$:
\begin{eqnarray}
\bzero = \cO(\bx_n + \Delta \bx_n) &\approx& \cO(\bx_n) + \bnabla \cO(\bx_n)\cdot\Delta\bx_n, \\
\Rightarrow \bnabla \cO(\bx_n) \cdot \Delta \bx_n &=& -\cO(\bx_n), \\
\bx_{n+1} &=& \bx_{n} + \Delta \bx_n. 
\end{eqnarray}

Each step requires the computation of the residual vector, $\cO(\bx)$, and
the gradient matrix of the operator, $\bnabla \cO(\bx)$, 
given the current solution $\bx = \bx_n$.
Then, a sparse linear system needs to be solved to yield the update for the 
increment $\Delta \bx_n = \bx_{n+1} - \bx_n$.

When the method converges, $\bx_n \rightarrow \bx_\infty$, 
the convergence is in general quadratic:
\begin{eqnarray}
\|\bx_{n+1} - \bx_\infty\| \le k \|\bx_{n} - \bx_\infty\|^2.
\end{eqnarray}
Thus, when the initial guess is close enough to the solution, the solver
will typically converge in very few steps.

\subsubsection{Operator Representation}
System variable $\bx$ is defined as the concatenation $[\,\varPhi, C, V_r, V_\theta, P\,]$,
taking only values in the interior of the problem domain grid.
Thus, each problem variable can be computed by applying an appropriate projection operator on $\bx$:
\begin{eqnarray}
X &=& \cP_X(\bx) \mbox{ for } X \in \left\{\varPhi, C, V_r, V_\theta, P\right\}.
\end{eqnarray}

The extended variables $\tilde X$ (including ``ghost'' point values) 
are computed by applying the following nonlinear operators:
\begin{eqnarray}
\tilde{\varPhi} &=& \cB_\varPhi(\varPhi, C; \beta), \\
\tilde{C} &=& \cB_C(C, \varPhi), \\
\tilde{V_r} &=& \cB_{V_r}(V_r; \cU), \\
\tilde{V_\theta} &=& \cB_{V_\theta}(V_\theta, C, \varPhi; \cU, \gamma), \\
\tilde{P} &=& P.
\end{eqnarray}
The extended system variable can be written as $\tilde{\bx} = \cB(\bx)$, 
where $\cB$ is the ``ghost'' point extension operator (derived using boundary conditions)
so the concatenated equations of the system (as described above) can be written 
as a non-linear operator $\cE(\tilde\bx) = \bzero$ acting on the extended system variable 
$\tilde{\bx}$, where $\cE$ is the conservation equations operator 
(derived by concatenation of the system equations).

The problem can be cast as the following non-linear system:
\begin{eqnarray}
\bzero = \cE (\tilde{\bx}) = \cE (\cB (\bx)) = \cO(\bx)
\end{eqnarray}
 where $\cO = \cE \circ \cB$.

\subsubsection{Steady-state}
Assume that $\cU \ui$ is the velocity of the ion exchanger 
(given that the fluid is at rest far away from the particle).
After the convergence of the solver, $\cO(\bx_n) \rightarrow \bzero$, 
the ``ghost'' point values are updated using $\tilde\bx = \cB(\bx_\infty)$,
so  that the surface force $\boldsymbol{f} = \tT \cdot \bn$ acting 
on the ion-exchanger surface
can be computed by integration over the ion exchanger's surface $\mathcal S$.

Due to symmetry considerations, the total force $\bF$ is 
aligned with $\ui$ and it vanishes iff $\ui \cdot \bF = 0$.
\begin{eqnarray}
\bF &=& \oint_\mathcal{S} \boldsymbol{f}  dA = 
\oint_\mathcal{S} \tT \cdot \bnhat  dA. \\
F_\imath = \ui \cdot \bF &=& \oint_\mathcal{S} \ui \cdot \boldsymbol{f}  dA = 
\oint_\mathcal{S} \ui \cdot \tT \cdot \bnhat  dA = 
\int_0^\pi f_\imath(\theta) \cdot 2\pi \sin\theta d\theta ,
\\  
f_\imath &=& \ui \cdot \tT \cdot \brhat,
\\  
f_\imath &=& \pars{-P + 2\cD_r(V_r) + 
\frac{1}{2}\pars{\cD_r(\varPhi)}^2 - \frac{1}{2r^2}\pars{\cD_\theta(\varPhi)}^2}\cos\theta 
\\  
&& -\pars{\cD_r(V_\theta) - \frac{V_\theta}{r}
+ \frac{1}{r}\cD_r(\varPhi) \cD_{\theta}(\varPhi)}\sin\theta, \\
 \mathcal{S} &=& \{\br : \|\br\|_2 = 1\}.
\end{eqnarray}

The integral for $F_\imath$ is approximated by 1D numerical quadrature
(using the mid-point rule), to yield 
the total force $F_\imath(\cU)$ as a function of the drift velocity $\cU$.

Since the goal is to find steady-state solution, $F_\imath(\cU)$ 
is required to be zero --
and the appropriate $\cU$ is found by a simple 1D root-finding algorithm,
applied as an outer loop.

\subsubsection{Continuation}

The iterative solver above can be used to compute the steady-state solution for any given $\beta$.
For $\beta \ll 1$, the linear terms are the dominant ones, so the solution is linear in $\beta$ 
(as derived in \cite{yariv2010migration}), and the solver convergence is fast.

However, it is no longer true for $\beta \sim O(1)$, since the nonlinear terms become dominant
and the iterative solver may not converge at all.
In order to find a solution for such $\beta$, continuation method is used:
the solver is applied to a sequence of $\{\beta_i\}_{i=0}^n$ such that $\beta_0 = 0$,
$\beta_n = \beta$ and the solution $\bx_i$ for $\beta_i$ is used as the solver initializer
for the problem of $\beta_{i+1}$.

Given that $|\beta_{i+1} - \beta_i| < \eps$ for small enough $\eps$, 
the next solution $\bx_{i+1}$ is close to the previous one $\bx_i$,
so the iterative solver converges well.

%%%%%%%%%%%%%%%%%%%%%%%%%%%%%%%%%%%%%%%%%%%%%%%%%%%%%%%%%%%%%%%%%%%%%%%%%%%%%%%%
\subsection{Solver Implementation}

Object-Oriented Design methodology is used for implementation.
The MATLAB programming language is chosen due to strong numerical capabilities
and high-level language features.

\subsubsection{Operator Interface $ \rightarrow \cO$}
The base \verb|Operator| interface is defined by 
supporting the computation of a residual vector \verb|op.res()|
and a gradient matrix \verb|op.grad()|, given an input vector $\bx$.

This interface is implemented by specific operators classes,
which are used to constuct the operator $\cO = \cE \circ \cB$, so that 
the Newton method solver (that requires computing the residual and the gradient) 
can be applied automatically.

Each such operator may have other operators as its inputs, so the residual
and the gradient are computed recursively, using the chain rule.

\paragraph{Linear operator}
If $\cO$ is linear, it can be represented by a matrix $L$, such that:
\begin{eqnarray}
\cO(\bx) &=& L \bx, \\
\bnabla\cO &=& L. 
\end{eqnarray}
Note that finite difference $\cD$, interpolation $\cI$ and 
upwind selection $\mathcal{U}$ operators
can be implemented as linear sparse operators, by constructing 
an appropriate sparse matrix $L$, having $O(\dim \bx)$ non-zeroes).

\paragraph{Pointwise scalar function}
Let $f: \R \rightarrow \R$ be a differentiable 1D function, whose derivative is
denoted by $f': \R \rightarrow \R$. The operator $\cF$ can be defined to
represent pointwise application of $f$:
\begin{eqnarray}
\cF(\bx) &=& [f(x_1); \ldots; f(x_n)], \\
\bnabla\cF(\bx) &=& \diag\{f'(x_1), \ldots, f'(x_n)\}.
\end{eqnarray}
$f$ is defined and differentiated automatically using the MATLAB symbolic toolbox.

\paragraph{Constant value}
A constant operator $\mathcal{C}$ is defined as having constant vector value $\bc$ 
(that does not depend on $\bx$):
\begin{eqnarray}
\mathcal{C}(\bx) &=& \bc, \\
\bnabla\mathcal{C} &=& \bzero.
\end{eqnarray}

\paragraph{Binary operators}
Let $\cO_1$ and $\cO_2$ be two operators, so their pointwise addition, subtraction,
and multiplication, are defined as follows:
\begin{eqnarray}
(\cO_1 \pm \cO_2)(\bx) &=& \cO_1)(\bx) \pm \cO_2(\bx), \\
\bnabla (\cO_1 \pm \cO_2)(\bx)   &=& \bnabla\cO_1(\bx) \pm \bnabla\cO_2(\bx), \\
(\cO_1 \cdot \cO_2)(\bx) &=& \diag(\diag(\cO_1(\bx)) \cdot \diag(\cO_2(\bx))), \\
\bnabla (\cO_1 \cdot \cO_2)(\bx) &=& \diag(\cO_2(\bx)) \bnabla\cO_1(\bx) 
                                  + \diag(\cO_1(\bx)) \bnabla\cO_2(\bx).
\end{eqnarray}

\paragraph{N-ary operators}
Given $N$ operators $\{\cO_i\}_{i=1}^N$, their concatenation $\cO$ is defined as:
\begin{eqnarray}
\cO(\bx) &=& [\cO_1(\bx); \ldots; \cO_N(\bx)], \\
(\bnabla\cO)(\bx) &=& [(\bnabla\cO_1)(\bx); \ldots; (\bnabla\cO_N)(\bx)].
\end{eqnarray}

\subsubsection{Iterative Newton solver $\rightarrow \bx$}
Given an initial solution $\bx_0$ and specific values for $\beta$ and $\cU$, 
we apply the following algorithm 
until convergence ($\|\Delta \bx\| < \eps$):
\begin{enumerate}
\item Update the system: $\verb|op| = \verb|System|(\bx_{n})$
\item Compute the gradient: $\boldsymbol{A}_n = \verb|op.grad()|$
\item Compute the right-hand side: $\boldsymbol{b}_n = \verb|-op.res()|$
\item Solve sparse linear system: $\boldsymbol{A}_n \Delta \bx_n = \boldsymbol{b}_n$
\item Update solution: $\bx_{n+1} = \bx_{n} + \Delta \bx_{n}$
\end{enumerate}

\subsubsection{Total force integration $\rightarrow F_\imath$}
After the iterative Newton solver converges ($\bx_n \rightarrow \bs$), 
the ``ghost'' point values are computed by computing $\cB(\bs)$ and
the total force $\bF$ acting on the ion-exchanger.

$F_\imath = \bF \cdot \ui$ is computed by using the mid-point quadrature method:
\begin{eqnarray}
F_\imath &=& \sum_{i=1}^{n_\theta} f(\bar\theta_i) \cdot 
              2 \pi \sin\bar\theta_i \cdot \Delta\theta_i
\end{eqnarray}

\subsubsection{Steady state drift velocity $\rightarrow \cU$}
In steady state, the total force acting on the particle $F_\imath$ must vanish.
This constraint yields an equation, whose solution is the steady-state velocity $\cU$:
\begin{eqnarray}
F_\imath(\cU) &=& 0
\end{eqnarray}
This equation can be solved using a 1D root finding algorithm, e.g. the secant method:
\begin{eqnarray}
\cU_{n+1} = \frac{F_\imath(\cU_{n})\cU_{n-1} - F_\imath(\cU_{n-1})\cU_{n}}
{F_\imath(\cU_{n}) - F_\imath(\cU_{n-1})} \rightarrow \cU
\end{eqnarray}
The result $\cU$ is the 
steady state particle drift velocity for the electric field $\beta$.

