\documentclass[10pt]{article}
\hsize=5.5 true in
\textheight=8.4 true in
\topmargin 1in

%% please put your definitions into mydef.sty. Examples can be found in this
%% file.
\usepackage{graphicx}
\usepackage{tikz}
\usepackage{amsmath} 
\usepackage{amssymb}
\usepackage{amsthm}
\usepackage{units}   
\usepackage{float}   
\usepackage{graphicx}
\usepackage{amsfonts}
\usepackage{mathrsfs}
\usepackage[colorlinks]{hyperref}
\usepackage{framed}
\usepackage{url}

\floatstyle{ruled}
\newfloat{program}{thp}{lop}
%-------------------------------------------
\newcommand{\sign}{\ensuremath{\mathrm{sign}}}
\newcommand{\diag}{\ensuremath{\mathrm{diag}}}
\newcommand{\trace}[1]{\ensuremath{\mathrm{trace}\left( #1 \right)}}
\newcommand{\norm}[1]{\ensuremath{\left\|#1\right\|_2^2}}
\newcommand{\func}[2]{\ensuremath{\mathrm{#1}\left( #2 \right)}}
\newcommand\eps \epsilon
\newcommand{\R}{\ensuremath{\mathbb{R}}}
\newcommand{\N}{\ensuremath{\mathbb{N}}}
\newcommand{\real}{\ensuremath{\mathbb{R}}}
\newcommand{\cl}[1]{\ensuremath{\mathcal{#1}}}
\newcommand{\suppsize}[1]{\ensuremath{|\mathcal{#1}|}}
\newcommand{\vect}[1]{\ensuremath{\mathbf{#1}}}
\newcommand{\matr}[1]{\ensuremath{\mathbf{#1}}}
\newcommand{\deriv}[2]{\frac{\partial #1}{\partial #2}}
\newcommand{\arr}[2]{\begin{array}{#1}#2\end{array}}
\newcommand{\mat}[2]{\left(\begin{array}{#1}#2\end{array}\right)}
\newcommand{\brc}[2]{\left\{\begin{array}{#1}#2\end{array}\right.}
\newcommand{\pars}[1]{\left(#1\right)}
\newcommand{\brcs}[1]{\left\{#1\right\}}
\newcommand{\half}{\frac{1}{2}}

% Bold symbols and operators
\newcommand\Laplacian{\nabla^2}
\newcommand\bnabla{\boldsymbol{\nabla}}
\newcommand\bLaplacian{\boldsymbol{\nabla}^2}
\newcommand\bcdot{\boldsymbol{\cdot}}
\newcommand\bU{\mathscr{\boldsymbol{U}}}
\newcommand\bv{\boldsymbol{v}}
\newcommand\bV{\boldsymbol{V}}
\newcommand\bE{\boldsymbol{E}}
\newcommand\be{\boldsymbol{\hat{e}}}
\newcommand\bn{\boldsymbol{\hat{n}}}
\newcommand\bk{\boldsymbol{\hat{k}}}
\newcommand\bj{\boldsymbol{j}}
\newcommand\bi{\boldsymbol{i}}
\newcommand\bs{\boldsymbol{s}}
\newcommand\bA{\boldsymbol{A}}
\newcommand\bF{\boldsymbol{F}}
\newcommand\bG{\boldsymbol{G}}
\newcommand\bI{\boldsymbol{I}}
\newcommand\bJ{\boldsymbol{J}}
\newcommand\bx{\boldsymbol{x}}
\newcommand\by{\boldsymbol{y}}
\newcommand\bz{\boldsymbol{z}}
\newcommand\br{\boldsymbol{r}}
\newcommand\bc{\boldsymbol{c}}
\newcommand\bxhat{\hat{\bx}}
\newcommand\byhat{\hat{\by}}
\newcommand\bzhat{\hat{\bz}}
\newcommand\brhat{\hat{\br}}
\newcommand\bnhat{\hat{\boldsymbol{n}}}
\newcommand\btheta{\boldsymbol{\theta}}
\newcommand\bthetahat{\hat{\btheta}}
\newcommand\bphi{\boldsymbol{\phi}}
\newcommand\bphihat{\hat{\bphi}}
\newcommand\bzero{\boldsymbol{0}}
\newcommand\bomega{\boldsymbol{\omega}}
\newcommand\bpsi{\boldsymbol{\psi}}

% Calligraphic symbols
\newcommand\cB{\mathcal{B}}
\newcommand\cE{\mathcal{E}}
\newcommand\cF{\mathcal{F}}
\newcommand\cO{\mathcal{O}}
\newcommand\cG{\mathcal{G}}
\newcommand\cI{\mathcal{I}}
\newcommand\cP{\mathcal{P}}
\newcommand\cD{\mathcal{D}}
\newcommand\cL{\mathcal{L}}
\newcommand\cU{\mathscr{U}}
\newcommand\cV{\mathscr{V}}
\newcommand\cW{\mathscr{W}}

% Tensors
\newcommand\tI{\mathsf{I}}
\newcommand\tS{\mathsf{S}}
\newcommand\tT{\mathsf{T}}

% Unit vector
\newcommand\ui{\boldsymbol{\hat{\imath}}}

\setlength{\textheight}{8.7in}
\setlength{\columnsep}{2.0pc}
\setlength{\textwidth}{6.6in}
% \setlength{\footheight}{0.2in}
\setlength{\topmargin}{0.05in}
\setlength{\headheight}{0.2in}
\setlength{\headsep}{0.1in}
\setlength{\evensidemargin}{0in}
\setlength{\oddsidemargin}{0in}
% \setlength{\parindent}{1pc}
\setlength{\parindent}{0.0 in}
\setlength{\parskip}{0.1 in}



%\usepackage{mydef}

%%%%%%%%%%%%%%%
\begin{document}
%%%%%%%%%%%%%%%%

\title{Asymptotic Nonlinear Solution for 
Electrokinetic Flow Around an Ion-Exchange Spherical Particle}

% author and address
\author{Roman Zeyde \and Irad Yavneh}

\maketitle

\begin{abstract}
An asymptotic nonlinear solution of electrokinetic 
migration of ion-exchange particles in an electrolyte solution 
due to the application of an external electric field is presented.
The electrokinetic transport process is
described by a system of nonlinear partial differential equations (PDE).
An asymptotic analytical solution is 
derived and validated. 
\end{abstract}

\section{Introduction} \label{sec:intro}

Electrokinetic theory describes the dynamics of charged particles
in ionic fluids \cite{masliyah2005book,kirby2010book}.
When a particle acquires surface charge, a layer
of ions of opposite charge is attracted to the surface via    
electric forces, creating a double-layer structure around the
particle. This structure, called
``Debye layer'', electrically screens the surface charge,
creating a potential difference between the particle and the outer
layer of the fluid bulk.
In cases where the layer width is much smaller than the particle
size, an analytical asymptotic solution for the Debye layer
dynamics can be derived \cite{yariv2010asymptotic},
yielding ``effective'' boundary conditions near
the particle surface for the ``outer'' bulk region equations.

The variables of the electrokinetic flow equations are the electrostatic
potential $\varPhi$, fluid velocity $\bV$ and its pressure $P$, and
ionic concentration $C$.
The boundary conditions are determined by the specific
problem under consideration and defined by the particle's
geometry, chemical characteristics, and the fluid dynamics.
The partial differential equations that describe the system dynamics
under an external electric field, are coupled and nonlinear, and
in general they have no analytic solution. 

A closed-form linear asymptotic solution has been developed for
spherical particles and weak electric field in an
axisymmetric setting \cite{yariv2010migration}. Once the electric field
becomes stronger, significant nonlinear phenomena emerge.
This interesting regime has not yet been explored extensively.

The goal of this work is to develop the next nonlinear terms
in the asymptotic solution of the ion-exchanger migration.
This solution can be used to gain insight into the chemical and physical behavior 
in far more general regimes than are currently well-understood, 
and also for numerical solver results verification.

The derivation is performed as follows. The governing differential equations are
\begin{equation} \begin{array}{cccc}
\bnabla \cdot \pars{ C \bnabla \varPhi } = 0, &
\Laplacian C - \alpha \bV \cdot \bnabla C = 0, &
\Laplacian \bV - \bnabla P + \bnabla \varPhi \Laplacian \varPhi = \bzero, &
\bnabla \cdot \bV = 0.
\end{array}\end{equation}
The boundary conditions on the particle surface are
\begin{equation}
\begin{array}{cccc}
0 = \varPhi + \log C, &
0 = \deriv{}{n} \pars{\varPhi - \log C}, &
\bV = \zeta \cdot \bnabla_\mathcal{S} \varPhi 
+ 2\log\pars{1-\tanh^2\frac{\zeta}{4}} \cdot \bnabla_\mathcal{S} \log C, &
\zeta = \log \frac{C}{\gamma}
\end{array}\end{equation}
Far away from the particle, the boundary conditions are
\begin{equation}\label{eq:bndcond_inf}\begin{array}{ccc}
\bV \rightarrow -\cU \ui, &
\bnabla \varPhi \rightarrow -\beta\ui, &
 C \rightarrow 1.
\end{array}\end{equation}

The system can be written as $\cO(\bx;\beta) = \bzero$.
For small $\beta$, the solution $\bx = [\varPhi; C; \bV; P]$ 
to the nonlinear system can be expanded in a Taylor series in $\beta$:
\begin{equation}
\bx(\beta) \approx \sum_n \bx_n \beta^n.
\end{equation}
The nonlinear terms
are expanded around $\beta = 0$:
\begin{equation}
\bzero = \cO(\bx) = \sum_i \cO_i(\bx_0, \ldots \bx_i) \beta^i.
\end{equation}

Thus, the $O(\beta^k)$ term $\bx_k$ can be found recursively by solving 
$\cO_k(\bx_0, \bx_1, \ldots, \bx_k) = \bzero$,
given the previous solutions for $\bx_0, \ldots, \bx_{k-1}$.


%%%%%%%%%%%%%%%%%%%%%%%%%%%%%%%%%%%%%%%%%%%%%%%%%%%%%%%%%%%%%%%%%%%%%%%%%%%
\section{First-Order (linear in $\beta$) Solution} \label{app:linear}

For $\beta = 0$ (no electric field is applied), the steady-state solution is $\cU = 0$:
\begin{equation}\begin{array}{cccc}
\varPhi_0(r,\theta) = 0, &
C_0(r,\theta) = 1, &
\bV_0(r,\theta) = \bzero, &
P_0(r,\theta) = 0.
\end{array}\end{equation}

Assuming $\beta \ll 1$, the linearized equations and boundary conditions are:
\begin{equation} \begin{array}{ccc}
\Laplacian \varPhi_1 = 0, &
\Laplacian C_1 = 0, &
\bLaplacian \bV_1 - \bnabla P_1 = \bzero.
\end{array}\end{equation}

The first-order terms $\bx_1$ are given by (see \cite{yariv2010migration} for full derivation):
\begin{equation} \begin{array}{ccccc}
\varPhi = \beta \pars{\frac{1}{4r^2} - r}\cos\theta, &
C = 1 + \beta \frac{3}{4r^2} \cos\theta, &
\Psi = \beta \cU_1 \pars{\frac{1}{r} - r^2} \frac{\sin^2\theta}{2}, &
P = 0, &
\bF = \bzero,
\end{array} \end{equation}
and the linear velocity term is:
\begin{equation} \label{eq:linear_velocity}
\cU_1(\gamma) = 2 \log \pars{\frac{1 + \gamma^{-\frac{1}{2}}}{2}}.
\end{equation}
For $\gamma > 1$, the Debye layer has a positive space charge, 
corresponding to a negative particle charge -- 
yielding negative drift velocity, $\cU < 0$. 
For $\gamma < 1$, we have $\cU > 0$.

%%%%%%%%%%%%%%%%%%%%%%%%%%%%%%%%%%%%%%%%%%%%%%%%%%%%%%%%%%%%%%%%%%%%%%%%%%%
\section{Second-Order (quadratic in $\beta$) Solution} \label{app:quadratic}
The quadratic terms $\bx_2$ are computed by solving a linear PDE system 
with a right-hand side determined by the linear terms $\bx_1$:
\begin{equation}
\varPhi_2 = \pars{\frac{\cU_1\, \alpha}{32} - \frac{1}{16}}
\frac{3\, {\cos^2\theta} - 1}{r^3} - \frac{3 + 3\, {\sin^2\theta}\, \left(4\, r^3 - 1\right)}{32\, r^4} - \frac{3\, \cU_1\, \alpha - 6}{32\, r},
\end{equation}
\begin{equation}
C_2 = \left(\frac{5\, \cU_1\, \alpha}{32} + \frac{1}{16}\right)
\frac{3\, {\cos^2\theta} - 1}{r^3} - \frac{3\, \cU_1\, \alpha - 6}{32\, r} + \frac{3\, \cU_1\, \alpha\, \left(2\, r^3\, {\sin^2\theta} + {\sin^2\theta} - 1\right)}{16\, r^4},
\end{equation}
\begin{equation}
\Psi_2 = \pars{\frac{9}{16(\sqrt{\gamma}+1)} - \frac{3}{16} \cU_1 (\cU_1 \alpha + 1)}
 \left(\frac{1}{r^2} - 1\right) \sin^2\theta \cos\theta.
\end{equation}
Note that the quadratic term does not contribute to the total force and staedy-state 
velocity terms, but it does change the fluid flow, 
the electric potential and the ionic concentration.

\section{Third-Order (cubic in $\beta$) Solution} \label{app:cubic}
Note that, due to symmetry considerations, $\cU(\beta)$ is an anti-symmetric function.
Therefore, $\cU_2 = 0$ and the next velocity term is the cubic one:
\begin{equation} \label{eq:cubic}
\cU(\beta) \approx \beta \cU_1 + \beta^3 \cU_3 + O(\beta^5).
\end{equation}
The cubic terms $\bx_3$ are computed by solving a linear PDE system 
with a right-hand side determined by the linear terms $\bx_1$ and the quadratic terms $\bx_2$.
\begin{align}
\varPhi_3 &= \frac{15\cU_1\alpha + 6 }{64}\cos\theta  - \frac{3 \cU_1\alpha}{32} {\cos}^3\theta
 + \frac{183 {\cU_1}^2\alpha^2 - 839 \cU_1\alpha - 470}{2560 r^2} \cos\theta 
\\ \nonumber &
+ \frac{3 \cos\theta \left(5 \cU_1\alpha - 6 {\cos^2\theta} - 4 \cU_1\alpha {\cos^2\theta} + 4\right)}{64 r^3} + \frac{\left(2 \cU_1\alpha - 1\right) \left(\cos\theta - 3 {\cos}^3\theta\right)}{64 r^5} 
+ {{\cos}^3\theta \left(\frac{15 \cU_1\alpha}{64} + \frac{3}{32}\right)}{r^{-2}} 
\\ \nonumber &
- \left(3 \cos\theta - 5 {\cos}^3\theta\right) \left( - \frac{19 {\cU_1}^2\alpha^2}{5120} + \frac{97 \cU_1\alpha}{5120} + \frac{3 \cU_2\alpha}{320} + \frac{21}{1280}\right){r^{-4}} 
% \\ \nonumber &
+ {{\cos}^3\theta \left(\frac{\cU_1\alpha}{64} + \frac{3}{64}\right)}{r^{-6}},
\end{align}
\begin{align}
C_3 &=
\frac{3 {{\cU_1}}^2\alpha^2 {\cos^3\theta}}{32} + \frac{\cos\theta \left(797 {{\cU_1}}^2\alpha^2 + 419 {\cU_1}\alpha - 512 {\cU_2}\alpha + 174\right)}{2560 r^2} 
- \frac{3 {\cU_1}\alpha \cos\theta \left(5 {\cU_1}\alpha + 2\right)}{64} 
\\ \nonumber &
- {\left(3 \cos\theta - 5 {\cos^3\theta}\right) \left(\frac{59 {{\cU_1}}^2\alpha^2}{5120} + \frac{103 {\cU_1}\alpha}{5120} + \frac{69 {\cU_2}\alpha}{320} + \frac{3}{1280}\right)}{r^{-4}} 
+ \frac{\alpha \left(\cos\theta - 3 {\cos^3\theta}\right) \left(5\alpha {{\cU_1}}^2 + 2 {\cU_1} + 16 {\cU_2}\right)}{128 r^5} 
\\ \nonumber &
- \frac{3\alpha \cos\theta \left( - 8\alpha {{\cU_1}}^2 {\cos^2\theta} + 7\alpha {{\cU_1}}^2 + 2 {\cU_1} + 32 {\cU_2} {\cos^2\theta} - 32 {\cU_2}\right)}{128 r^3} 
%\\ \nonumber &
+ \frac{{{\cU_1}}^2\alpha^2 {\cos^3\theta}}{32 r^6} - \frac{3 {\cU_1}\alpha {\cos^3\theta} \left(5 {\cU_1}\alpha + 2\right)}{64 r^2},
\end{align}
\begin{align}
\Psi_3 &=
r^2 {\sin^2\theta} \left(\frac{{\cW_2}}{15} - \frac{{\cW_1}}{3} + \frac{209}{3360}\right) 
 - {{\sin^2\theta} \left(\frac{5 {\cW_2}}{3} - \frac{{\cW_1}}{3} + \frac{35}{264}\right)}{r^{-1}}
+ \frac{{\sin^4\theta} \left(2 {\cW_2} + \frac{761}{5632}\right)}{r} 
\\ \nonumber & 
- \frac{{\sin^2\theta} \left(1848 r^6 {\sin^2\theta} - 2079 r^3 {\sin^2\theta} + 924 r^3 - 7 {\sin^2\theta} + 10\right)}{19712 r^5} 
%\\ \nonumber & 
\frac{4 {\sin^2\theta} - 5 {\sin^4\theta}}{r^3} + \left(\frac{{\cU_1}\alpha}{16} - \frac{1}{8}\right) \frac{2 r^3 - 3 r^2 + 1}{4 r} {\sin^2\theta},
\end{align}
\begin{align}
\cW_1 &= \frac{69}{512 \left(\sqrt{\gamma} + 1\right)} - \frac{1407\alpha^2 {\log\left(\frac{\sqrt{\gamma} + 1}{2 \sqrt{\gamma}}\right)}^3}{1280} - \frac{123 \log\left(\frac{\sqrt{\gamma} + 1}{2 \sqrt{\gamma}}\right)}{1280} - \frac{27}{512 {\left(\sqrt{\gamma} + 1\right)}^2} 
-\frac{1899\alpha {\log\left(\frac{\sqrt{\gamma} + 1}{2 \sqrt{\gamma}}\right)}^2}{2560} 
\\ \nonumber & 
- \frac{\alpha \log\left(\frac{\sqrt{\gamma} + 1}{2 \sqrt{\gamma}}\right) \left(\frac{81}{16 \left(\sqrt{\gamma} + 1\right)} - \log\left(\frac{\sqrt{\gamma} + 1}{2 \sqrt{\gamma}}\right) \left(\frac{27\alpha \log\left(\frac{\sqrt{\gamma} + 1}{2 \sqrt{\gamma}}\right)}{4} + \frac{27}{8}\right)\right)}{320} 
+ \frac{21\alpha \log\left(\frac{\sqrt{\gamma} + 1}{2 \sqrt{\gamma}}\right)}{128 \left(\sqrt{\gamma} + 1\right)},
\end{align}
\begin{align}
\cW_2 &= \frac{9 \log\left(\frac{\sqrt{\gamma} + 1}{2 \sqrt{\gamma}}\right)}{256} 
+ \frac{21\alpha^2 {\log\left(\frac{\sqrt{\gamma} + 1}{2 \sqrt{\gamma}}\right)}^3 + 
15\alpha {\log\left(\frac{\sqrt{\gamma} + 1}{2 \sqrt{\gamma}}\right)}^2}{64} 
- \frac{27}{512 {\left(\sqrt{\gamma} + 1\right)}^2}
%\\ \nonumber &
- \frac{297\alpha \log\left(\frac{\sqrt{\gamma} + 1}{2 \sqrt{\gamma}}\right) + 54}{1024 \left(\sqrt{\gamma} + 1\right)}.
\end{align}

The cubic solution satisfies the ${r\rightarrow\infty}$ boundary conditions, 
when $\alpha\cU_1 = 0$.

\section{Validation}
The derivation above is validated using MATLAB symbolic toolbox by the code 
at \verb|symbolic/asymp.m| \cite{source}.

\bibliographystyle{unsrt}
\bibliography{ijnam}

\end{document}
