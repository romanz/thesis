\documentclass[10pt]{ijnam}
\hsize=5.5 true in
\textheight=8.4 true in
\topmargin 1in

%% information for the editor
\def\currentvolume{1}
\def\currentissue{1}
\def\currentyear{2004}
\def\month{March}
\pagespan{1}{18}
\copyrightinfo{2004}{} % copyright year

%% please put your definitions into mydef.sty. Examples can be found in this
%% file.


%\usepackage{mydef}

%%%%%%%%%%%%%%%
\begin{document}
%%%%%%%%%%%%%%%%

\title[Sample for how to use ijnam.cls]
{Enter the title of your paper here\\
  If it is too long, you can use \\
  to break it into several lines}

% author and address
\author[Short Name]{Full Name}
\address{
  Department of Mathematics,
  University of Newland,
  Newland, NL 88888, USA
}
\email{hyang@louisiana.edu}
\urladdr{http://www.ucs.lousiana.edu/$\sim$hxy6615/}

% two authors have the same address
\author[S. Name 1\and S. Name 2]{Full Name 1 \and Full Name 2}
\address{
  Department of Mathematics,
  University of Newland,
  Newland, NL 88888, USA
}
\email{yang@louisiana.edu \and hong@louisiana.edu}
\urladdr{http://www.ucs.lousiana.edu/$\sim$hxy6615/ \and
  http://www.ucs.lousiana.edu/$\sim$hxy6615/}

% dedication
\dedicatory{This paper is dedicated to our authors}

% communicated
\commby{Hilbert}

% Received by the editors ?
\date{January 1, 2004 and, in revised form, March 22, 2004.}

% it is suggested to put it in Acknowledgments
\thanks{This research was supported by ???}

\subjclass[2000]{35R35, 49J40, 60G40}

\abstract{This paper is a sample prepared to illustrate the use of
ijnam.cls. You can use all commands of amsart.cls.}

\keywords{point, line, plane, and space.}

\maketitle

\section{First section}
This is the first section.

\section{Second section}

This is the second section.

\subsection{This section has two subsections}

This is the first subsection of the second section.

\subsection{The second subsection}

This is the first subsection of the second section.
\begin{equation}
c=a+b.
\end{equation}
This equation can rewritten as:
\begin{equation}\label{eq2}
a=c-b.
\end{equation}

\subsubsection{The second subsection has two subsubsections}

This is the first subsubsection of the second subsection.

\subsubsection{The second subsubsection}

This is the second subsubsection of the second subsection.

For $b=1$, $c=4$, we get from (\ref{eq2})
\begin{equation}
a=3.
\end{equation}

%%%%%%%%%%%%%%%%%%%%%%%
\section{Forth section}
%%%%%%%%%%%%%%%%%%%%%%%

This is the forth section.

%%%%%%%%%%%%%%%%%%%%%%%
\section{Third section}
%%%%%%%%%%%%%%%%%%%%%%%

This is the last section.

% put your thanks here
\section*{Acknowledgments}
The author thanks the anonymous authors whose work largely
constitutes this sample file. This research was supported by ???.

\begin{thebibliography}{99}

\bibitem{frie} Friedman, Avner,
Partial Differential Equations of Parabolic Type , Robert E. Krieger
Publishing Company, Malabar, Florida, 1983.

\bibitem {glow} Glowinski, R., Lions, J.~L. and Tr\'{e}moli\`{e}res, R.,
Numerical Analysis of Variational Inequalities, Norh-Holland Publishing
Company, 1981.

\bibitem{lsu} Ladyzenskaya, O.A., Solonnikov, V.A. and Uralceva, N.N.,
Linear and Quasi-linear Equations of Parabolic Type, AMS, 1988.

\bibitem{lions} Lions, J.L. and Magenes, E.,
Non-Homogeneous Boundary Value Problems and Applications I-II,
Springer-Verlag, New York, 1972.

\end{thebibliography}

\end{document}
